\documentclass{beamer}
\usepackage{graphicx}
% \usepackage{beamerthemesplit} // Activate for custom appearance

\title{Comparison of ALICE data with the Jet-Jet Monte-Carlo Simulation}
\author{Ivan Chernyshev}
\date{\today}

\begin{document}

\frame{\titlepage}

\section[Outline]{}
%\frame{\tableofcontents}

%\section{Introduction}
%\subsection{Overview of the Beamer Class}
\frame
{
  \frametitle{What aspects of the simulation and the data were compared}

  \begin{itemize}
  \item The mass-pion data and its fits
  \item The shape of the normalized transverse momentum $\pi^0$ spectra 
  \item The measured (mean) values of the $\pi^0$ mass and the widths (standard deviations) of the $\pi^0$ peak in the data 
  \item The cutflow of the $\gamma$ data    
  \end{itemize}
}

\frame
{
	\frametitle{Mass-Pion Data Comparison: 6-8 GeV}
	\noindent\hspace{1.5 cm}ALICE data: 
	\noindent\hspace{3.0 cm} Simulation data:\\
	\includegraphics[width=0.55\textwidth]{{Pion_data/THnSparses_072417_output/gaussian/data/angle_17mrad/MyFit_Ptmin_6.00_Ptmax_8.00}.png}
	\includegraphics[width=0.55\textwidth]{{monte_carlo/background/Pion_data/gaussian/quadratic_my_code/data/angle_17mrad/MyFit_Ptmin_6.00_Ptmax_8.00}.png}
}

\frame
{
	\frametitle{Mass-Pion Data Comparison: 8-10 GeV}
	\noindent\hspace{1.5 cm}ALICE data: 
	\noindent\hspace{3.0 cm} Simulation data:\\
	\includegraphics[width=0.55\textwidth]{{Pion_data/THnSparses_072417_output/gaussian/data/angle_17mrad/MyFit_Ptmin_8.00_Ptmax_10.00}.png}
	\includegraphics[width=0.55\textwidth]{{monte_carlo/background/Pion_data/gaussian/quadratic_my_code/data/angle_17mrad/MyFit_Ptmin_8.00_Ptmax_10.00}.png}
}

\frame
{
	\frametitle{Mass-Pion Data Comparison: 10-12 GeV}
	\noindent\hspace{1.5 cm}ALICE data: 
	\noindent\hspace{3.0 cm} Simulation data:\\
	\includegraphics[width=0.5\textwidth]{{Pion_data/THnSparses_072417_output/gaussian/data/angle_17mrad/MyFit_Ptmin_10.00_Ptmax_12.00}.png}
	\includegraphics[width=0.5\textwidth]{{monte_carlo/background/Pion_data/gaussian/quadratic_my_code/data/angle_17mrad/MyFit_Ptmin_10.00_Ptmax_12.00}.png}
}

\frame
{
	\frametitle{Mass-Pion Data Comparison: 12-14 GeV}
	\noindent\hspace{1.5 cm}ALICE data: 
	\noindent\hspace{3.0 cm} Simulation data:\\
	\includegraphics[width=0.5\textwidth]{{Pion_data/THnSparses_072417_output/gaussian/data/angle_17mrad/MyFit_Ptmin_12.00_Ptmax_14.00}.png}
	\includegraphics[width=0.5\textwidth]{{monte_carlo/background/Pion_data/gaussian/quadratic_my_code/data/angle_17mrad/MyFit_Ptmin_6.00_Ptmax_8.00}.png}
}

\frame
{
	\frametitle{Mass-Pion Data Comparison: 14-16 GeV}
	\noindent\hspace{1.5 cm}ALICE data: 
	\noindent\hspace{3.0 cm} Simulation data:\\
	\includegraphics[width=0.5\textwidth]{{Pion_data/THnSparses_072417_output/gaussian/data/angle_17mrad/MyFit_Ptmin_14.00_Ptmax_16.00}.png}
	\includegraphics[width=0.5\textwidth]{{monte_carlo/background/Pion_data/gaussian/quadratic_my_code/data/angle_17mrad/MyFit_Ptmin_14.00_Ptmax_16.00}.png}
}

\frame
{
	\frametitle{Measured $\pi^0$ Masses Comparison: Raw Data}
	\includegraphics[width=1.0\textwidth]{combination/Pion_data/MeanMasses.png}	
}

\frame
{
	\frametitle{Measured $\pi^0$ Masses Comparison: Ratios}
	\includegraphics[width=1.0\textwidth]{combination/Pion_data/PionMassRatios.png}	
}

\frame
{
	\frametitle{Measured $\pi^0$ Mass Peak Widths Comparison: Raw Data}
	\includegraphics[width=1.0\textwidth]{combination/Pion_data/MassWidths.png}	
}

\frame
{
	\frametitle{Measured $\pi^0$ Mass Peak Widths Comparison: Ratios}
	\includegraphics[width=1.0\textwidth]{combination/Pion_data/PionMassWidthRatios.png}	
}

\frame 
{ 
\frametitle{Analysis of Cuts: Pt 10-50 GeV, Jet-Jet Monte-Carlo simulation} 
\begin{table} 
\caption{How cuts affect number of photons} 
\centering 
\begin{tabular}{c c c c} 
\hline\hline 
Cuts & Percentage of Previous\\ [0.5ex] 
\hline
None & n/a\\
+$0.1 < \lambda <$ 0.4 & 29.0 $\%$ \\
+dR $> 20$ mrad & 54.7 $\%$ \\
+DisToBorder $>$ 0 & 92.5 $\%$ \\
+DisToBadCell $>$ 1 & 74.6 $\%$ \\
+Ncells $> 1$ & 100.0 $\%$ \\
+Exoticity $< 0.97$ & 99.9 $\%$ \\
+$|Time| < 30$ ns & n/a (time is not well-modeled)  \\
+isolation $< 4 GeV/c$ & 21.4 $\%$ \\
[1ex] 
\hline 
\end{tabular} 
\label{table:nonlin} 
\end{table} 
 Final value's portion of the original: 2.3\%
 } 

\frame 
{ 
\frametitle{Analysis of Cuts: Pt 10-50 GeV, ALICE data } 
\begin{table} 
\caption{How cuts affect number of photons} 
\centering 
\begin{tabular}{c c c c} 
\hline\hline 
Cuts & Percentage of Previous\\ [0.5ex] 
\hline
None & n/a\\
+$0.1 < \lambda <$ 0.4 & 57.9 $\%$ \\
+dR $> 20$ mrad & 88.8 $\%$ \\
+DisToBorder $>$ 0 & 92.8 $\%$ \\
+DisToBadCell $>$ 1 & 78.4 $\%$ \\
+Ncells $> 1$ & 100.0 $\%$ \\
+Exoticity $< 0.97$ & 99.9 $\%$ \\
+$|Time| < 30$ ns & 99.8 $\%$ \\
+isolation $< 4 GeV/c$ & 56.9 $\%$ \\
[1ex] 
\hline 
\end{tabular} 
\label{table:nonlin} 
\end{table} 
 Final value's portion of the original: 21.2\%
 } 

\frame
{
	\frametitle{ALICE data: $\gamma$ spectrum ratios of various distance to border and distance to bad cell bins}
	\includegraphics[width=0.55\textwidth]{Direct_Photon_data/cut_comparisons/0.1<lambda<0.4/4<pT<20/photon_ratios_distoborder.png}
	\includegraphics[width=0.55\textwidth]{Direct_Photon_data/cut_comparisons/0.1<lambda<0.4/4<pT<20/photon_ratios_distobadcells.png}
}

\frame
{
	\frametitle{Simulation data: $\gamma$ spectrum ratios of various distance to border and distance to bad cell bins}
	\includegraphics[width=0.55\textwidth]{monte_carlo/background/Direct_Photon_data/cut_comparisons/0.1<lambda<0.4/5<pT<20/photon_ratios_distoborder.png}
	\includegraphics[width=0.55\textwidth]{monte_carlo/background/Direct_Photon_data/cut_comparisons/0.1<lambda<0.4/5<pT<20/photon_ratios_distobadcells.png}
}

\frame
{
	\frametitle{Shape of $\pi^0$-pT Spectrum Comparison}
	\includegraphics[width=1.0\textwidth]{combination/Pion_data/NumOfPionsGraph.png}	
}

\end{document}
