\documentclass{beamer}
\usepackage{graphicx}
% \usepackage{beamerthemesplit} // Activate for custom appearance

\title{Pion Data Analysis in Non-Gaussian Peak Models}
\author{Ivan Chernyshev}
\date{\today}

\begin{document}

\frame{\titlepage}

\section[Outline]{}
\frame{\tableofcontents}

\section{Overview of the Data}
\subsection{Immediately after my previous presentation, I compared different models of the data. I briefly touched on alternative models of the background, and settled on the quadric model. I then did an analysis on various models of the peak and then settled upon the Crystal Ball Function. However, the number of pions seemed to be dipping too low, and Miguel suspected that the cuts on the data may be responsible. So I did a few tests on the cuts to the data, and found that only the lambda and angle cuts are relevant, and the lambda cut is indeed responsible for much of the pion loss. I then plotted the energies of the leading photon with respect to that of the trailing photon.
}
\frame
{
  \frametitle{Alternative Models of the Background (Entries over Mass)}
Quadric: 
\noindent\hspace{2.5 cm}Damped sine ($\frac{sin(x)}{x^a}$):
\noindent\hspace{1.0 cm}Power Law:\\
\includegraphics[width=0.35\textwidth]{old/Different_models/Quadric/mass_pion_plot.png}
\includegraphics[width=0.35\textwidth]{old/Different_models/Sine/mass_pion_plot.png}
\includegraphics[width=0.35\textwidth]{old/Different_models/Power/mass_pion_plot.png}\\
Logarithmic:  
\noindent\hspace{1.5 cm}Logistic:\\
\includegraphics[width=0.35\textwidth]{old/Different_models/Logarithmic/mass_pion_plot.png}
\includegraphics[width=0.35\textwidth]{old/Different_models/Logistic/mass_pion_plot.png}\\
I was going to work on the error in the logarithmic and logistic models but then I realized that working on alternative peak fits was more important
}
\frame
{
\frametitle{Damped Sine and Quadric model comparison}
The best two background models, as evidenced from the residuals in the previous slide, are the damped sine and the quadric model. Here, I will compare the two models' mass vs. entries graphs for different momentum intervals.\\
Quadric: 8-10 GeV/c:
\noindent\hspace{1.0 cm} 10-11 GeV/c:
\noindent\hspace{1.0 cm} 11-12 GeV/c:
\includegraphics[width=0.35\textwidth]{{old/Different_models/Quadric/MyFit_Ptmin_8.00_Ptmax_10.00}.png}
\includegraphics[width=0.35\textwidth]{{old/Different_models/Quadric/MyFit_Ptmin_10.00_Ptmax_11.00}.png}
\includegraphics[width=0.35\textwidth]{{old/Different_models/Quadric/MyFit_Ptmin_11.00_Ptmax_12.00}.png}\\
Damped Sine: 8-10 GeV/c: 
\noindent\hspace{0.75 cm} 10-11 GeV/c:
\noindent\hspace{0.75 cm} 11-12 GeV/c:
\includegraphics[width=0.35\textwidth]{{old/Different_models/Sine/MyFit_Ptmin_8.00_Ptmax_10.00}.png}
\includegraphics[width=0.35\textwidth]{{old/Different_models/Sine/MyFit_Ptmin_10.00_Ptmax_11.00}.png}
\includegraphics[width=0.35\textwidth]{{old/Different_models/Sine/MyFit_Ptmin_11.00_Ptmax_12.00}.png}
}

\frame
{
Quadric: 12-13 GeV/c:
\noindent\hspace{1.5 cm} 13-15 GeV/c:
\includegraphics[width=0.35\textwidth]{{old/Different_models/Quadric/MyFit_Ptmin_12.00_Ptmax_13.00}.png}
\includegraphics[width=0.35\textwidth]{{old/Different_models/Quadric/MyFit_Ptmin_13.00_Ptmax_15.00}.png}\\
Damped Sine: 12-13 GeV/c:
\noindent\hspace{1.0 cm} 13-15 GeV/c:
\includegraphics[width=0.35\textwidth]{{old/Different_models/Sine/MyFit_Ptmin_12.00_Ptmax_13.00}.png}
\includegraphics[width=0.35\textwidth]{{old/Different_models/Sine/MyFit_Ptmin_13.00_Ptmax_15.00}.png}\\
Conclusion: the damped sine model, as evidenced from the residuals, is better than the quadric model by a noticeable amount for the entire sample. However, there is no noticeable difference for the individual momentum intervals, and the damped sine model introduces the groundless assumption that the background is sinusoidal, so I continued using the quadric model.
}

\frame 
{
  \frametitle{Alternative Models of the Peak(Entries over Mass)}
3 candidates:\\
Crystal Ball: 
\noindent\hspace{2.0 cm}Gaussian:
\noindent\hspace{2.0cm}ExpGaussian:\\
\includegraphics[width=0.35\textwidth]{old/Different_models/data/Crystal_Ball/mass_pion_plot.png}
\includegraphics[width=0.35\textwidth]{old/Different_models/data/Gaussian/mass_pion_plot.png}
\includegraphics[width=0.35\textwidth]{old/Different_models/data/Exponential_Gaussian/mass_pion_plot.png}\\
Rejected:\\
Skew Normal:
\noindent\hspace{1.5 cm} Rayleigh curve:\\
\includegraphics[width=0.35\textwidth]{old/Different_models/Skew_Normal/mass_pion_plot.png}
\includegraphics[width=0.35\textwidth]{old/Different_models/Modified_Rayleigh/mass_pion_plot.png} 
}

\frame
{
 \frametitle{Comparison of Peak Models: Entries over mass}
 Red: Gaussian Model; Blue: ExpGaussian model; Green: Crystal Ball Model\\
 8-10 GeV:
 \noindent\hspace{2.0 cm}10-11 GeV:
 \noindent\hspace{2.0 cm}11-12 GeV:\\
 \includegraphics[width=0.35\textwidth]{{old/Different_models/data_comparisons/testgraph_8.00GeV-10.00GeV}.png}
 \includegraphics[width=0.35\textwidth]{{old/Different_models/data_comparisons/testgraph_10.00GeV-11.00GeV}.png}
 \includegraphics[width=0.35\textwidth]{{old/Different_models/data_comparisons/testgraph_11.00GeV-12.00GeV}.png}\\
 12-13 GeV:
 \noindent\hspace{2.0 cm}13-15 GeV:
 \includegraphics[width=0.35\textwidth]{{old/Different_models/data_comparisons/testgraph_12.00GeV-13.00GeV}.png}
 \includegraphics[width=0.35\textwidth]{{old/Different_models/data_comparisons/testgraph_13.00GeV-15.00GeV}.png}\\
 From the residuals and the fits, it is evident that the Crystal Ball function is the best, by a slight amount
}

\frame
{
\frametitle{Signal-to-noise ratio}
Gaussian Model:\\
\includegraphics[width=0.5\textwidth]{old/Different_models/data/Gaussian/Overall_Signal_Over_Total.png}
\includegraphics[width=0.5\textwidth]{old/Different_models/data/Exponential_Gaussian/Overall_Signal_Over_Total.png}\\
\includegraphics[width=0.5\textwidth]{old/Different_models/data/Crystal_Ball/Overall_Signal_Over_Total.png}
}

\frame 
{
\frametitle{Comparison of Peak Models: Parameters}
Mean masses over momentum: 
\noindent\hspace{1.0 cm} Mass standard deviations:
\includegraphics[width=0.5\textwidth]{old/Different_models/data_comparisons/MassvsMomentum.png}
\includegraphics[width=0.5\textwidth]{old/Different_models/data_comparisons/MassWidthvsMomentum.png}\\
Number of pions:\\
\includegraphics[width=0.5\textwidth]{old/Different_models/data_comparisons/PeakIntegralvsMomentum.png}\\
}

\frame 
{
\frametitle{Analysis of cuts}
The number of pions is too low here, so I decided to investigate if one of the cuts may have been responsible. In addition to the cuts I investigated previously, 
I also added in a new cut: the number of matched tracks must be zero, which is the number for photons, which $\pi^0$ decays into, and photons are how we are finding the $\pi^0$.\\
Here are the matched tracks vs entries charts (Note that the software automatically gives a result with zero matched tracks a value of -1):\\
\includegraphics[width=0.5\textwidth]{data/0cuts/MatchedTracks1_pion_plot.png}
\includegraphics[width=0.5\textwidth]{data/0cuts/MatchedTracks2_pion_plot.png}
}

\frame
{
\frametitle{Analysis of cuts: Mass vs. Entries}
Whole range of momenta:
8-10 GeV:
\noindent\hspace{1.0 cm} 10-11 GeV:\\
\includegraphics[width=0.35\textwidth]{data_comparisons/testgraph.png}
\includegraphics[width=0.35\textwidth]{{data_comparisons/testgraph_8.00GeV-10.00GeV}.png}
\includegraphics[width=0.35\textwidth]{{data_comparisons/testgraph_10.00GeV-11.00GeV}.png}\\
11-12 GeV:
\noindent\hspace{1.5 cm} 12-13 GeV:
\noindent\hspace{1.5 cm} 13-15 GeV:\\
\includegraphics[width=0.35\textwidth]{{data_comparisons/testgraph_11.00GeV-12.00GeV}.png}
\includegraphics[width=0.35\textwidth]{{data_comparisons/testgraph_12.00GeV-13.00GeV}.png}
\includegraphics[width=0.35\textwidth]{{data_comparisons/testgraph_13.00GeV-15.00GeV}.png}
}

\frame
{
\frametitle{Analysis of cuts: Signal-to-noise ratio}
No cuts:
\noindent\hspace{2.0 cm}Matched track cut:
\noindent\hspace{0.5 cm}Matched track and\\ 
\noindent\hspace{8.0 cm}asymmetry cut:\\
\includegraphics[width=0.35\textwidth]{data/0cuts/Overall_Signal_Over_Total.png}
\includegraphics[width=0.35\textwidth]{data/1cuts/Overall_Signal_Over_Total.png}
\includegraphics[width=0.35\textwidth]{data/2cuts/Overall_Signal_Over_Total.png}\\
Matched track, asymmetry, 
All cuts except for $\lambda_{02}$:
\noindent\hspace{1.0 cm}All cuts:\\
and angle cut:\\
\includegraphics[width=0.35\textwidth]{data/3cuts/Overall_Signal_Over_Total.png}
\includegraphics[width=0.35\textwidth]{data/4cuts/Overall_Signal_Over_Total.png}
\includegraphics[width=0.35\textwidth]{data/5cuts/Overall_Signal_Over_Total.png}
}

\frame
{
\frametitle{Analysis of cuts: Parameters}
Mean masses over momentum: 
\noindent\hspace{1.0 cm} Mass standard deviations:
\includegraphics[width=0.5\textwidth]{data_comparisons/MassvsMomentum.png}
\includegraphics[width=0.5\textwidth]{data_comparisons/MassWidthvsMomentum.png}\\
Number of pions:\\
\includegraphics[width=0.5\textwidth]{data_comparisons/PeakIntegralvsMomentum.png}\\
}

\end{document}

