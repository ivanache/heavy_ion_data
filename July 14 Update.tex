\documentclass{beamer}
\usepackage{graphicx}
\usepackage{animate}
\title{Update on $\gamma/\pi^0$ data cut analysis}
\author{Ivan Chernyshev}
\date{\today}

\begin{document}
\frame{\titlepage}

%\section[Outline]{}
%\frame{\tableofcontents}

%\section{Introduction}
%\subsection{Overview of the Beamer Class}

\begin{frame}
  \frametitle{Effects of distance to border and distance to bad cells cuts}
  \framesubtitle{(Courtesy of Miguel Arratia)}
  \animategraphics[loop,controls,width=\linewidth]{1}{effectofbadchannelcut-}{0}{6}
\end{frame}

\frame
{
  \frametitle{Effects of the distance to border and distance to bad cells cuts}
  Since the distance to border and distance to bad cell cuts cut out significant amounts of data, I want to know at which distance we should draw the line 
  \begin{itemize}
  \item Applied all baseline cuts, except for the distance to border and distance to bad cell
  \item Plotted the normalized transverse momentum spectra of the photons that have various distances to the bad cell and border
  \item Quantified the difference in spectral shape between the data near the border or a bad cell and the rest of the data
  %\item Plotted the ratios along with the theoretical fit that the spectrum ratios =1 
  %\item Found the chi-square and p-value for the set of ratios for each spectrum
  %\item Repeated the whole process for a distance to border of 0, 1, 2, and $>2$, respectively
  \end{itemize}
}

\frame 
{
    \frametitle{Photon Spectrum Charts}
    \framesubtitle{pT 4-20 GeV, 0.1$<\lambda_{02}<$0.4}
    \includegraphics[width=0.5\textwidth]{{Direct_Photon_data/cut_comparisons/0.4<lambda/4<pT<20/pt_photon_distobadcell}.png}
    \includegraphics[width=0.5\textwidth]{{Direct_Photon_data/cut_comparisons/0.4<lambda/4<pT<20/pt_photon_distoborder}.png}
}

\frame
{
  \frametitle{Effects of the distance to bad cells cut}
  \framesubtitle{pT 4-20 GeV, 0.1$<\lambda_{02}<$0.4}
 % Distance to Bad Cells Cut
  %\noindent\hspace{2.0 cm} Distance to Border Cut\\
   \includegraphics[width=1.0\textwidth]{{Direct_Photon_data/cut_comparisons/0.1<lambda<0.4/4<pT<20/photon_ratios_distobadcells}.png}\\
   %\includegraphics[width=0.5\textwidth]{{Direct_Photon_data/cut_comparisons/0.4<lambda/4<pT<20/photon_ratios_distobadcells}.png}\\
   %\includegraphics[width=0.5\textwidth]{{Direct_Photon_data/cut_comparisons/0.4<lambda/4<pT<20/photon_ratios_distoborder}.png}\\
   Based on the p-values, only the photons with a distance to bad cell = 1 should be cut
}

\frame
{
  \frametitle{Effects of the distance to border cut}
  \framesubtitle{pT 4-20 GeV, 0.1$<\lambda_{02}<$0.4}
   \includegraphics[width=1.0\textwidth]{{Direct_Photon_data/cut_comparisons/0.1<lambda<0.4/4<pT<20/photon_ratios_distoborder}.png}\\
   Based on the p-values, only the photons with a distance to border = 0 should be cut
}

\frame
{
   \frametitle{Distance to bad cell cut effect, $\lambda_{02} > $0.4}
   \framesubtitle{pT 4-20 GeV}
   \includegraphics[width=1.0\textwidth]{{Direct_Photon_data/cut_comparisons/0.4<lambda/4<pT<20/photon_ratios_distobadcells}.png}
   Above  $\lambda_{02} > $0.4, the ratio is far worse for distance to bad cells = 1, and for the rest the ratios are better
}

\frame
{
   \frametitle{Distance to border cut effect, $\lambda_{02} > $0.4}
   \framesubtitle{pT 4-20 GeV}
   \includegraphics[width=1.0\textwidth]{{Direct_Photon_data/cut_comparisons/0.4<lambda/4<pT<20/photon_ratios_distoborder}.png}
   Above  $\lambda_{02} > $0.4, the ratios are better across the board.
}


\frame 
{
    \frametitle{Photon Cutflow}
    \begin{itemize}
    \item Next, I graphed the non-normalized photon spectrum for a data sample with no cuts, then counted the number of photons in the spectrum between 10 and 20 GeV.\\
    \item I repeated this six times, adding a baseline cut each time, in order to get a cutflow.
    \end{itemize} 
}

\frame
{
    \frametitle{Photon Cutflow: pT 10-20 GeV}
     \begin{table} 
     \caption{How cuts affect number of photons} 
     \centering 
     \begin{tabular}{c c c c} 
     \hline\hline 
    Cuts & Num of Photons\\ [0.5ex] 
    \hline
    None & 15251\\
    +$0.1 < \lambda <$ 0.4 & 8937\\
    +dR $> 20$ mrad & 7217\\
    +DisToBorder $>$ 0 & 6690\\
    +DisToBadCell $>$ 1 & 4950\\
    +Ncells $> 1$ & 4950\\
    +Exoticity $< 0.97$ & 4948\\
    +$|Time| < 30$ ns & 4944\\
    [1ex] 
    \hline 
    \end{tabular} 
    \label{table:nonlin} 
    \end{table} 
}

\frame
{
	 \frametitle{Photon Cutflow: $\lambda_{02}$ vs Photons}
	 \includegraphics[width=1.0\textwidth]{Direct_Photon_data/modeler/lambda_photon.png}
}

\frame
{
      \frametitle{Photon Cutflow: Distance to Charged Particle (dR) vs Photons}
      \includegraphics[width=1.0\textwidth]{Direct_Photon_data/modeler/distance_to_charged_photon.png}
}

\frame
{
      \frametitle{Photon Cutflow: Distance to Border vs Photons}
      \includegraphics[width=1.0\textwidth]{Direct_Photon_data/modeler/distance_to_border_photon.png}
}

\frame
{
      \frametitle{Photon Cutflow: Distance to Bad Cells vs Photons}
      \includegraphics[width=1.0\textwidth]{Direct_Photon_data/modeler/distance_to_bad_cells_photon.png}
}

\frame
{
      \frametitle{Photon Cutflow: Ncells vs Photons}
      \includegraphics[width=1.0\textwidth]{Direct_Photon_data/modeler/number_of_cells_photon.png}
}

\frame
{
      \frametitle{Photon Cutflow: Exoticity vs Photons}
      \includegraphics[width=1.0\textwidth]{Direct_Photon_data/modeler/exoticity_photon.png}
}

\frame
{
      \frametitle{Photon Cutflow: Cluster Time vs Photons}
      \includegraphics[width=1.0\textwidth]{Direct_Photon_data/modeler/time_photon.png}
}

\frame
{
	\frametitle{Summary}
	\begin{itemize}
	\item By quantitatively comparing the data from different cuts, I found that distance to bad cells should be cut to greater than 1 and the distance to border, to greater than 0
	\item The $\lambda_{02}$ cut is by far the costliest cut in terms of number of particles, so for small data samples it may be useful to forego the use of this cut
	\item The dR, distance to border, and distance to bad cell cuts all have a moderate effect on the number of photons, but the other three cuts have barely any effect
	\end{itemize}
}

% Reserve Slides
\frame
{
  \frametitle{Effects of the distance to bad cells cut}
 % Distance to Bad Cells Cut
  %\noindent\hspace{2.0 cm} Distance to Border Cut\\
   \includegraphics[width=1.0\textwidth]{{Direct_Photon_data/cut_comparisons/0.1<lambda<0.4/4<pT<20/pt_photon_distobadcell}.png}
   %\includegraphics[width=0.5\textwidth]{{Direct_Photon_data/cut_comparisons/0.4<lambda/4<pT<20/pt_photon_distobadcell}.png}\\
   %\includegraphics[width=0.5\textwidth]{{Direct_Photon_data/cut_comparisons/0.4<lambda/4<pT<20/pt_photon_distoborder}.png}\\ 
}

\frame
{
  \frametitle{Effects of the distance to border cut}
   \includegraphics[width=1.0\textwidth]{{Direct_Photon_data/cut_comparisons/0.1<lambda<0.4/4<pT<20/pt_photon_distoborder}.png}
}


\end{document}
