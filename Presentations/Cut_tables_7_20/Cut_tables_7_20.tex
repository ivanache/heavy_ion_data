\documentclass{beamer}
\usepackage{graphicx}
% \usepackage{beamerthemesplit} // Activate for custom appearance

\title{Cut Data Tables}
\author{Ivan Chernyshev}
\date{\today}

\begin{document}

%\frame
%{
%\frametitle{Analysis of Cuts: Old Data Table; Pt 8-15 GeV}
%\begin{table}
%\caption{How cuts affect data quality}
%\centering
%\begin{tabular}{c c c c}
%\hline\hline
%Cuts & \# Pions1 & S/T at 1 $\sigma$ &  S/T at 2 $\sigma$ \\ [0.5ex] % inserts table %heading
%\hline
%None & 11003& 0.75 & 0.69 \\
%+M.Tracks & 7798 & 0.82 & 0.76 \\
%+asymmetry & 7798 & 0.82 & 0.76 \\
%+angle& 7414 & 0.79 & 0.72 \\
%+Ncells & 7456 & 0.79 & 0.73 \\
%+lambda & 4183 & 0.86 & 0.82 \\ [1ex]
%\hline
%\end{tabular}
%\label{table:nonlin}
%\end{table}
%}

% For both the 7-20 momentum interval and every sub-interval within the 7-20 interval, input the instructions from the corresponding file in Table_instructions

\frame
{
\frametitle{Analysis of Cuts: Pt 8-20 GeV}
\begin{table}
\caption{How cuts affect data quality}
\centering
\begin{tabular}{c c c c}
\hline\hline
Cuts & \# Pions1 & S/T at 1 $\sigma$ & S/T at 2 $\sigma$ \\ [0.5ex]
\hline
None & 49785 +/-  276 & 0.73 & 0.65 \\ %0%
+angle $> 15$ mrad & 38771 +/-  246 & 0.71 & 0.64 \\ %1%
+$0.1 < \lambda <$ 0.4 & 21345 +/-  164 & 0.85 & 0.80 \\ %2%
+dR $> 20$ mrad & 19647 +/-  155 & 0.88 & 0.83 \\ %3%
+asymmetry $< 0.7$ & 19638 +/-  227 & 0.88 & 0.83 \\ %4%
[1ex]
\hline
\end{tabular}
\label{table:nonlin}
\end{table}
}


\frame
{
\frametitle{Analysis of Cuts: Pt  8-10 GeV}
\begin{table}
\caption{How cuts affect data quality}
\centering
\begin{tabular}{c c c c}
\hline\hline
Cuts & \# Pions1 & S/T at 1 $\sigma$ & S/T at 2 $\sigma$ \\ [0.5ex]
\hline
None & 2221 +/-   57 & 0.85 & 0.81 \\ %0%
+dR $> 20$ mrad & 1278 +/-   41 & 0.84 & 0.78 \\ %1%
+asymmetry $< 0.7$ & 1278 +/-   41 & 0.84 & 0.78 \\ %2%
+angle $> 15$ mrad & 1017 +/-   36 & 0.84 & 0.79 \\ %3%
+Ncells $> 1$& 1044 +/-   36 & 0.87 & 0.82 \\ %4%
+DisToBorder $>$ 2 & 1134 +/-  408 & 0.94 & 0.93 \\ %5%
+DisToBadCell $>$ 1&  546 +/-   26 & 0.89 & 0.84 \\ %6%
+$0.1 < \lambda <$ 0.4 &  203 +/-   16 & 0.89 & 0.85 \\ %7%
[1ex]
\hline
\end{tabular}
\label{table:nonlin}
\end{table}
}


\frame
{
\frametitle{Analysis of Cuts: Pt 10-11 GeV}
\begin{table}
\caption{How cuts affect data quality}
\centering
\begin{tabular}{c c c c}
\hline\hline
Cuts & \# Pions1 & S/T at 1 $\sigma$ & S/T at 2 $\sigma$ \\ [0.5ex]
\hline
None &  846 +/-   48 & 0.82 & 0.76 \\ %0%
+dR $> 20$ mrad &  807 +/-   41 & 0.85 & 0.80 \\ %1%
+asymmetry $< 0.7$ &  803 +/-   41 & 0.85 & 0.80 \\ %2%
+angle $> 15$ mrad &  417 +/-   38 & 0.74 & 0.68 \\ %3%
+Ncells $> 1$&  403 +/-   24 & 0.75 & 0.68 \\ %4%
+DisToBorder $>$ 2 &  360 +/-   29 & 0.79 & 0.73 \\ %5%
option 1\\
+DisToBadCell $>$ 1&  265 +/-   23 & 0.83 & 0.77 \\ %6%
+$0.1 < \lambda <$ 0.4 &  129 +/-   13 & 0.85 & 0.80 \\ %7%
option 2\\
+DisToBadCell $>$ 2&  141 +/-   19 & 0.73 & 0.66 \\ %6%
+$0.1 < \lambda <$ 0.4 &   89 +/-   10 & 0.88 & 0.85 \\ %7%
[1ex]
\hline
\end{tabular}
\label{table:nonlin}
\end{table}
}


\frame
{
\frametitle{Analysis of Cuts: Pt 11-12 GeV}
\begin{table}
\caption{How cuts affect data quality}
\centering
\begin{tabular}{c c c c}
\hline\hline
Cuts & \# Pions1 & S/T at 1 $\sigma$ & S/T at 2 $\sigma$ \\ [0.5ex]
\hline
None & 2178 +/-  101 & 0.70 & 0.63 \\ %0%
+dR $> 20$ mrad & 2052 +/-   94 & 0.77 & 0.71 \\ %1%
+asymmetry $< 0.7$ & 2015 +/-   86 & 0.76 & 0.70 \\ %2%
+angle $> 15$ mrad & 1439 +/-   82 & 0.72 & 0.66 \\ %3%
+Ncells $> 1$& 1333 +/-   72 & 0.71 & 0.65 \\ %4%
+DisToBorder $>$ 2 & 1117 +/-   55 & 0.72 & 0.65 \\ %5%
+DisToBadCell $>$ 1&  762 +/-   43 & 0.73 & 0.66 \\ %6%
+$0.1 < \lambda <$ 0.4 &  485 +/-   40 & 0.88 & 0.84 \\ %7%
[1ex]
\hline
\end{tabular}
\label{table:nonlin}
\end{table}
}


\frame
{
\frametitle{Analysis of Cuts: Pt 12-13 GeV}
\begin{table}
\caption{How cuts affect data quality}
\centering
\begin{tabular}{c c c c}
\hline\hline
Cuts & \# Pions1 & S/T at 1 $\sigma$ & S/T at 2 $\sigma$ \\ [0.5ex]
\hline
None & 5827 +/-  209 & 0.60 & 0.52 \\ %0%
+dR $> 20$ mrad & 6199 +/-  201 & 0.72 & 0.66 \\ %1%
+asymmetry $< 0.7$ & 6123 +/-  173 & 0.72 & 0.65 \\ %2%
+angle $> 15$ mrad & 4846 +/-  156 & 0.72 & 0.65 \\ %3%
+Ncells $> 1$& 4297 +/-  157 & 0.70 & 0.63 \\ %4%
+DisToBorder $>$ 2 & 3598 +/-  136 & 0.70 & 0.63 \\ %5%
+DisToBadCell $>$ 1& 2730 +/-  106 & 0.75 & 0.69 \\ %6%
+$0.1 < \lambda <$ 0.4 & 1458 +/-   44 & 0.82 & 0.76 \\ %7%
[1ex]
\hline
\end{tabular}
\label{table:nonlin}
\end{table}
}


\frame
{
\frametitle{Analysis of Cuts: Pt 13-15 GeV}
\begin{table}
\caption{How cuts affect data quality}
\centering
\begin{tabular}{c c c c}
\hline\hline
Cuts & \# Pions1 & S/T at 1 $\sigma$ & S/T at 2 $\sigma$ \\ [0.5ex]
\hline
None & 11565 +/-  130 & 0.76 & 0.69 \\ %0%
+dR $> 20$ mrad & 10759 +/-  120 & 0.82 & 0.76 \\ %1%
+asymmetry $< 0.7$ & 10759 +/-  120 & 0.82 & 0.76 \\ %2%
+angle $> 15$ mrad & 8494 +/-  107 & 0.81 & 0.75 \\ %3%
+$0.1 < \lambda <$ 0.4 & 5009 +/-  102 & 0.93 & 0.91 \\ %4%
[1ex]
\hline
\end{tabular}
\label{table:nonlin}
\end{table}
}


\frame
{
\frametitle{Analysis of Cuts: Pt 15-20 GeV}
\begin{table}
\caption{How cuts affect data quality}
\centering
\begin{tabular}{c c c c}
\hline\hline
Cuts & \# Pions1 & S/T at 1 $\sigma$ & S/T at 2 $\sigma$ \\ [0.5ex]
\hline
None &  556 +/-   42 & 0.68 & 0.60 \\ %0%
+dR $> 20$ mrad &  554 +/-   28 & 0.75 & 0.68 \\ %1%
+asymmetry $< 0.7$ &  548 +/-   39 & 0.75 & 0.67 \\ %2%
+angle $> 15$ mrad &  210 +/-   39 & 0.68 & 0.60 \\ %3%
+Ncells $> 1$&  249 +/-  111 & 0.73 & 0.66 \\ %4%
+DisToBorder $>$ 2 &  150 +/-  160 & 0.68 & 0.60 \\ %5%
option 1\\
+DisToBadChannel $>$ 1&  111 +/-   13 & 0.69 & 0.61 \\ %6%
+$0.1 < \lambda <$ 0.4 &   81 +/-   25 & 0.85 & 0.81 \\ %7%
option 2\\
+DisToBadChannel $>$ 2&  118 +/-   46 & 0.80 & 0.75 \\ %6%
+$0.1 < \lambda <$ 0.4 &   62 +/-   14 & 0.90 & 0.89 \\ %7%
[1ex]
\hline
\end{tabular}
\label{table:nonlin}
\end{table}
}


\end{document}
