\documentclass{beamer}
\usepackage{graphicx}
% \usepackage{beamerthemesplit} // Activate for custom appearance

\title{Cut Data Tables}
\author{Ivan Chernyshev}
\date{\today}

\begin{document}

%\frame
%{
%\frametitle{Analysis of Cuts: Old Data Table; Pt 8-15 GeV}
%\begin{table}
%\caption{How cuts affect data quality}
%\centering
%\begin{tabular}{c c c c}
%\hline\hline
%Cuts & \# Pions1 & S/T at 1 $\sigma$ &  S/T at 2 $\sigma$ \\ [0.5ex] % inserts table %heading
%\hline
%None & 11003& 0.75 & 0.69 \\
%+M.Tracks & 7798 & 0.82 & 0.76 \\
%+asymmetry & 7798 & 0.82 & 0.76 \\
%+angle& 7414 & 0.79 & 0.72 \\
%+Ncells & 7456 & 0.79 & 0.73 \\
%+lambda & 4183 & 0.86 & 0.82 \\ [1ex]
%\hline
%\end{tabular}
%\label{table:nonlin}
%\end{table}
%}

% For both the 7-20 momentum interval and every sub-interval within the 7-20 interval, input the instructions from the corresponding file in Table_instructions

\documentclass{beamer} 
\usepackage{graphicx} 

\title{Cut Data Tables} 
\author{Ivan Chernyshev} 
\date{\today} 

\begin{document} 

\frame 
{ 
\frametitle{Analysis of Cuts: Pt 10-50 GeV, Monte-Carlo simulation} 
\begin{table} 
\caption{How cuts affect number of photons} 
\centering 
\begin{tabular}{c c c c} 
\hline\hline 
Cuts & Num of Photons & Percentage of Previous\\ [0.5ex] 
\hline
None & 2996161 & n/a\\
+$0.1 < \lambda <$ 0.4 & 869787 & 29.0 $\%$ \\
+dR $> 20$ mrad & 475600 & 54.7 $\%$ \\
+DisToBorder $>$ 0 & 440136 & 92.5 $\%$ \\
+DisToBadCell $>$ 1 & 328331 & 74.6 $\%$ \\
+Ncells $> 1$ & 328326 & 100.0 $\%$ \\
+Exoticity $< 0.97$ & 327939 & 99.9 $\%$ \\
+$|Time| < 30$ ns & n/a & (time is not well-modeled)  \\
+isolation $< 4 GeV/c$ & 70064 & 25.5 $\%$ \\
[1ex] 
\hline 
\end{tabular} 
\label{table:nonlin} 
\end{table} 
 Final value's portion of the original: 2.3\%
 } 
\end{document}

\frame
{
\frametitle{Analysis of Cuts: Pt  8-10 GeV}
Note: asymmetry without cuts is always less than 1
\begin{table}
\caption{How asymmetry cuts affect data quality}
\centering
\begin{tabular}{c c c c c}
\hline\hline
Asymmetry Cut & \# Pions1 & S/T at 1 $\sigma$ & S/T at 2 $\sigma$ & p-val \\ [0.5ex]
\hline
asym $<$ 1 & 3300 +/-   95 & 0.81 & 0.74 & 0.698 \\ %1%
asym $<$ 0.9 & 3300 +/-   95 & 0.81 & 0.74 & 0.698 \\ %0.9%
asym $<$ 0.8 & 3158 +/-   91 & 0.83 & 0.77 & 0.446 \\ %0.8%
asym $<$ 0.7 & 2641 +/-   80 & 0.84 & 0.79 & 0.714 \\ %0.7%
asym $<$ 0.6 & 2132 +/-   70 & 0.84 & 0.79 & 0.791 \\ %0.6%
[1ex]
\hline
\end{tabular}
\label{table:nonlin}
\end{table}
}


\frame
{
\frametitle{Analysis of Cuts: Pt 10-11 GeV}
\begin{table}
\caption{How cuts affect data quality}
\centering
\begin{tabular}{c c c c}
\hline\hline
Cuts & \# Pions1 & S/T at 1 $\sigma$ & S/T at 2 $\sigma$ \\ [0.5ex]
\hline
None & 7531 +/-  110 & 0.70 & 0.62 \\ %0%
+dR $> 20$ mrad & 6805 +/-   99 & 0.77 & 0.70 \\ %1%
+asymmetry $< 0.7$ & 6792 +/-   99 & 0.77 & 0.70 \\ %2%
+angle $> 15$ mrad & 5440 +/-  203 & 0.76 & 0.70 \\ %3%
+Ncells $> 1$& 4899 +/-   84 & 0.76 & 0.69 \\ %4%
+DisToBorder $>$ 2 & 4192 +/-   77 & 0.77 & 0.71 \\ %5%
+DisToBadCell $>$ 1& 3062 +/-  122 & 0.81 & 0.75 \\ %6%
+$0.1 < \lambda <$ 0.4 & 1611 +/-   45 & 0.87 & 0.82 \\ %7%
[1ex]
\hline
\end{tabular}
\label{table:nonlin}
\end{table}
}


\frame
{
\frametitle{Analysis of Cuts: Pt 11-12 GeV}
\begin{table}
\caption{How cuts affect data quality}
\centering
\begin{tabular}{c c c c}
\hline\hline
Cuts & \# Pions1 & S/T at 1 $\sigma$ & S/T at 2 $\sigma$ \\ [0.5ex]
\hline
None &  950 +/-   59 & 0.79 & 0.73 \\ %0%
+dR $> 20$ mrad &  896 +/-   44 & 0.82 & 0.77 \\ %1%
+asymmetry $< 0.7$ &  894 +/-   44 & 0.82 & 0.77 \\ %2%
+angle $> 15$ mrad &  585 +/-   39 & 0.84 & 0.79 \\ %3%
+Ncells $> 1$&  575 +/-   36 & 0.87 & 0.82 \\ %4%
+DisToBorder $>$ 2 &  481 +/-   31 & 0.88 & 0.84 \\ %5%
+$0.1 < \lambda <$ 0.4 &  260 +/-   17 & 0.92 & 0.90 \\ %6%
[1ex]
\hline
\end{tabular}
\label{table:nonlin}
\end{table}
}


\frame
{
\frametitle{Analysis of Cuts: Pt 12-13 GeV}
\begin{table}
\caption{How cuts affect data quality}
\centering
\begin{tabular}{c c c c}
\hline\hline
Cuts & \# Pions1 & S/T at 1 $\sigma$ & S/T at 2 $\sigma$ \\ [0.5ex]
\hline
None & 6301 +/-  199 & 0.63 & 0.55 \\ %0%
+dR $> 20$ mrad & 6020 +/-  178 & 0.71 & 0.64 \\ %1%
+asymmetry $< 0.7$ & 6306 +/-   98 & 0.73 & 0.67 \\ %2%
+angle $> 15$ mrad & 4892 +/-  107 & 0.72 & 0.65 \\ %3%
+Ncells $> 1$& 4453 +/-  125 & 0.72 & 0.65 \\ %4%
+DisToBorder $>$ 2 & 3747 +/-  123 & 0.72 & 0.65 \\ %5%
+DisToBadCell $>$ 1& 2828 +/-  275 & 0.75 & 0.69 \\ %6%
+$0.1 < \lambda <$ 0.4 & 1475 +/-   53 & 0.82 & 0.77 \\ %7%
[1ex]
\hline
\end{tabular}
\label{table:nonlin}
\end{table}
}


\frame
{
\frametitle{Analysis of Cuts: Pt 13-15 GeV}
\begin{table}
\caption{How cuts affect data quality}
\centering
\begin{tabular}{c c c c}
\hline\hline
Cuts & \# Pions1 & S/T at 1 $\sigma$ & S/T at 2 $\sigma$ \\ [0.5ex]
\hline
None &  789 +/-  157 & 0.74 & 0.67 \\ %0%
+dR $> 20$ mrad &  773 +/-   33 & 0.78 & 0.72 \\ %1%
+asymmetry $< 0.7$ &  774 +/-   33 & 0.78 & 0.72 \\ %2%
+angle $> 15$ mrad &  367 +/-   24 & 0.70 & 0.63 \\ %3%
+Ncells $> 1$&  555 +/-   73 & 0.92 & 0.88 \\ %4%
+DisToBorder $>$ 2 &  351 +/-   35 & 0.87 & 0.82 \\ %5%
+DisToBadCell $>$ 1&  209 +/-   17 & 0.77 & 0.71 \\ %6%
+$0.1 < \lambda <$ 0.4 &  129 +/-   12 & 0.88 & 0.84 \\ %7%
[1ex]
\hline
\end{tabular}
\label{table:nonlin}
\end{table}
}


\frame
{
\frametitle{Analysis of Cuts: Pt 15-20 GeV}
\begin{table}
\caption{How cuts affect data quality}
\centering
\begin{tabular}{c c c c}
\hline\hline
Cuts & \# Pions1 & S/T at 1 $\sigma$ & S/T at 2 $\sigma$ \\ [0.5ex]
\hline
None &  692 +/-  323 & 0.72 & 0.66 \\ %0%
+dR $> 20$ mrad &  435 +/-   57 & 0.64 & 0.56 \\ %1%
+asymmetry $< 0.7$ &  432 +/-   27 & 0.65 & 0.57 \\ %2%
+angle $> 15$ mrad &  234 +/-   18 & 0.77 & 0.69 \\ %3%
+Ncells $> 1$&  218 +/-   30 & 0.73 & 0.65 \\ %4%
+DisToBorder $>$ 2 &  210 +/-   16 & 0.90 & 0.83 \\ %5%
+DisToBadCell $>$ 1&   98 +/-   13 & 0.65 & 0.57 \\ %6%
+$0.1 < \lambda <$ 0.4 &   60 +/-    9 & 0.77 & 0.70 \\ %7%
[1ex]
\hline
\end{tabular}
\label{table:nonlin}
\end{table}
}


\end{document}
