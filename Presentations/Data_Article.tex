\documentclass[11pt]{article}
\usepackage{graphicx}
\usepackage{geometry}
\geometry{letterpaper}
\usepackage[parfill]{parskip}    
\usepackage{amssymb}   
\title{Signal vs Total for Various Cuts}
\author{Ivan Chernyshev}
% Set up the file, import all necessary packages, add title and author name

\begin{document}
\maketitle

% This section contains the graphs for collisions vs. momentum, asymmetry, lambda01, lambda02, and angle alpha absolute value
\section{Collision data versus each parameter other than mass (which is reserved for every other section)}

\begin{frame}{} % Momentum
\centering
\textbf{Pion Entries over Momentum}\par\medskip
\includegraphics[width=0.7\textwidth]{0cuts/momentum_pion_plot.png}
\end{frame}

\begin{frame}{} % Asymmetry
\centering
\textbf{Pion Entries over Asymmetry}\par\medskip
\includegraphics[width=0.7\textwidth]{0cuts/asymmetry_pion_plot.png}
\end{frame}

\begin{frame}{} % Angle alpha
\centering
\textbf{Pion Entries over $\mid\alpha\mid$}\par\medskip
\includegraphics[width=0.7\textwidth]{0cuts/angle_pion_plot.png}
\end{frame}

\begin{frame}{} % Lambda_01
\centering
\textbf{Pion Entries over $\lambda_{01}$}\par\medskip
\includegraphics[width=0.7\textwidth]{0cuts/lambda1_pion_plot.png}
\end{frame}


\begin{frame}{} % Lambda_02
\centering
\textbf{Pion Entries over $\lambda_{02}$}\par\medskip
\includegraphics[width=0.7\textwidth]{0cuts/lambda2_pion_plot.png}
\end{frame}

\begin{frame}{} % Ncells_01
\centering
\textbf{Pion Entries over $Ncells_{01}$}\par\medskip
\includegraphics[width=0.7\textwidth]{0cuts/Ncells1_pion_plot.png}
\end{frame}

\begin{frame}{} % Ncells_02
\centering
\textbf{Pion Entries over $Ncells_{02}$}\par\medskip
\includegraphics[width=0.7\textwidth]{0cuts/Ncells2_pion_plot.png}
\end{frame}

\section{Collisions over Mass}
\subsection{Guide to the graphs} % A quick guide for how the graphs work
Each of these graphs measures the pion entries over mass of the data for a different permutation of cuts to the interval of some of the data's parameters, along with a fit to the data (red line) and the Gaussian component of the fit (blue line)

There are three parameters: $\lambda$, Asymmetry, and absolute value of angle $\alpha$, for which the data is either cut to a prespecified interval or is not cut. The prespecified intervals are: $\lambda < 0.4$, Asymmetry $< 0.7, \mid\alpha\mid > 0.015 $.
\subsection{Pion Entries over Mass for All Quantities of Cuts} % The section with the collisions vs. mass graphs for all possible permutations of cuts
\begin{frame}{}
No cuts at all:\\
\includegraphics[width=0.5\textwidth]{0cuts/mass_pion_plot.png}\\
$\lambda$ cuts: 
\noindent\hspace{3 cm} $\lambda_{02}$ and asymmetry cuts: \\
\includegraphics[width=0.5\textwidth]{1cuts/mass_pion_plot.png}
\includegraphics[width=0.5\textwidth]{2cuts/mass_pion_plot.png}
$\lambda$, asymmetry, and $\mid\alpha\mid$ cuts:
\noindent\hspace{4 cm} $\lambda$, asymmetry, $\mid\alpha\mid$, and Ncells cuts:\\
\includegraphics[width=0.5\textwidth]{3cuts/mass_pion_plot.png}
\includegraphics[width=0.5\textwidth]{4cuts/mass_pion_plot.png}
\end{frame}

\section{Reduced Chi-Square}

\subsection{Guide to the graphs} % A quick go-over the purpose of the graphs
The code that I have programmed creates mass-pion entry plots with fits for not the entire data permitted by the cuts, but also divides the data into various momentum intervals, namely for momentum 5-7.5 GeV, momentum 7.5-10 GeV, momentum 10-11 GeV, momentum 11-12 GeV, momentum 12-13 GeV, momentum 13-15 GeV, momentum 15-18 GeV.

These graphs will be included at the end of the article, because there are so many of them that it would be impractical  to show them here, in the middle of multiple sections. To make the goodness of the fits of the mass-pion plots for the individual momentum intervals easier to cram into a single space, the code also generates chi-square graphs for each permutation of cuts. These graphs are shown here.

The cut permutations are the same as in the previous sections.

\subsection{Reduced Chi-Square vs Momentum for  All Quantities of Cuts} % Shows all of the chi-square graphs
\begin{frame}{}
No cuts at all:\\
\includegraphics[width=0.5\textwidth]{0cuts/reduced_chisquare_v_pT.png}\\
$\lambda$ cuts: 
\noindent\hspace{3 cm} $\lambda_{02}$ and asymmetry cuts: \\
\includegraphics[width=0.5\textwidth]{1cuts/reduced_chisquare_v_pT.png}
\includegraphics[width=0.5\textwidth]{2cuts/reduced_chisquare_v_pT.png}
$\lambda$, asymmetry, and $\mid\alpha\mid$ cuts:
\noindent\hspace{4 cm} $\lambda$, asymmetry, $\mid\alpha\mid$, and Ncells cuts:\\
\includegraphics[width=0.5\textwidth]{3cuts/reduced_chisquare_v_pT.png}
\includegraphics[width=0.5\textwidth]{4cuts/reduced_chisquare_v_pT.png}
\end{frame}

% This section contains the data for signal-to-noise ratio
\section{Signal-to-Noise Ratio}
\subsection{Guide to the graphs} % A guide/legend for the graphs lying below this subsection
Each of these graphs measures the peak signal to total detected output for an interval centered at the peak's mean as a function of the interval's radius in standard deviations.

Each color represents a curve for various momentum intervals.\\
Red = Momentum 5-7.5 GeV, Blue = Momentum 7.5-10 GeV, Green = Momentum 10-11 GeV, \\
Yellow = Momentum 11-12 GeV, Cyan = Momentum 12-13 GeV, Magenta = Momentum 13-15 GeV, Black =  Momentum 15-18 GeV

The cut permutations are the same as in the previous sections.

\subsection{Signal/Total Ratio for All Quantities of Cuts} % The section with the signal/total ratio graphs
\begin{frame}{}
No cuts at all:\\
\includegraphics[width=0.5\textwidth]{0cuts/Overall_Signal_Over_Total.png}\\
$\lambda$ cuts: 
\noindent\hspace{3 cm} $\lambda_{02}$ and asymmetry cuts: \\
\includegraphics[width=0.5\textwidth]{1cuts/Overall_Signal_Over_Total.png}
\includegraphics[width=0.5\textwidth]{2cuts/Overall_Signal_Over_Total.png}
$\lambda$, asymmetry, and $\mid\alpha\mid$ cuts:
\noindent\hspace{4 cm} $\lambda$, asymmetry, $\mid\alpha\mid$, and Ncells cuts:\\
\includegraphics[width=0.5\textwidth]{3cuts/Overall_Signal_Over_Total.png}
\includegraphics[width=0.5\textwidth]{4cuts/Overall_Signal_Over_Total.png}
\end{frame}

% This section contains data for the correlation between measured pion momentum and measured pion mass
\section{Mass-momentum correlation graphs} 
\subsection{Guide to the graphs} % A quick guide to the data in the section
The code that I wrote uses the mean of the gaussian part of the fits of the data from each momentum interval's mass-collision plots to find the measured masses of the pions for each momentum interval. These measured masses are then graphed over momentum. The expected mass of a pion is given as a straight line.

Additionally, the widths of the gaussian parts of the fits, which are standard deviations , are also graphed against momentum in order to give the viewer a measure of the potential errors in the data.

From the mass vs. momentum data, the measured pion mass does not follow the theory that mass is constant. This confirms the data from the paper here: https://link.springer.com/article/10.1140/epjc/s10052-017-4890-x and suggests that there is a flaw in an experimental procedure.
% Next time refer to the paper more professionally, like with title and author

The graphs are given for each cut of data in exactly the same manner as in the previous sections 

\subsection{Mass vs. Momentum for All Quantities of Cuts}% The subsection with the mass vs momentum graphs
\begin{frame}{}
$\lambda$, asymmetry, $\mid\alpha\mid$, and Ncells cuts:\\
\includegraphics[width=0.5\textwidth]{4cuts/meanMass_v_pT.png}\\
$\lambda$, asymmetry, and $\mid\alpha\mid$ cuts:
\noindent\hspace{3 cm}$\lambda$ and asymmetry cuts:\\
\includegraphics[width=0.5\textwidth]{3cuts/meanMass_v_pT.png}
\includegraphics[width=0.5\textwidth]{2cuts/meanMass_v_pT.png}
$\lambda$ cuts:
\noindent\hspace{4 cm} No cuts at all:\\
\includegraphics[width=0.5\textwidth]{1cuts/meanMass_v_pT.png}
\includegraphics[width=0.5\textwidth]{0cuts/meanMass_v_pT.png}
\end{frame}

\subsection{Width vs. Momentum for All Quantities of Cuts}% The subsection with the width vs. momentum graphs
\begin{frame}{}
$\lambda$, asymmetry, $\mid\alpha\mid$, and Ncells cuts:\\
\includegraphics[width=0.5\textwidth]{4cuts/massWidths_v_pT.png}\\
$\lambda$, asymmetry, and $\mid\alpha\mid$ cuts:
\noindent\hspace{3 cm}$\lambda$ and asymmetry cuts:\\
\includegraphics[width=0.5\textwidth]{3cuts/massWidths_v_pT.png}
\includegraphics[width=0.5\textwidth]{2cuts/massWidths_v_pT.png}
$\lambda$ cuts:
\noindent\hspace{4 cm} No cuts at all:\\
\includegraphics[width=0.5\textwidth]{1cuts/massWidths_v_pT.png}
\includegraphics[width=0.5\textwidth]{0cuts/massWidths_v_pT.png}
\end{frame}

\section{Number of pions per Momentum Interval} % This section contains data for number of pions over momentum 
\subsection{Guide to the graphs} % A quick introduction to what the graphs mean
My code uses the integral of the Gaussian part of the collisions vs mass curve divided by the width of each bin of a collision vs. mass graph (the bin width is the same for all graphs) to calculate the number of pions per interval of momentum for each permutation of cuts. The graphs showing the number of pions in each momentum interval for each permutation of cuts are shown here.

The data for all of the permutations of cuts is shown in the same manner as in the above sections.

\subsection{Number of Pions vs. Momentum for All Quantities of Cuts} % The subsection with the number of pions vs. momentum graphs
\begin{frame}{}
$\lambda$, asymmetry, $\mid\alpha\mid$, and Ncells cuts:\\
\includegraphics[width=0.5\textwidth]{4cuts/peakIntegrals_v_pT.png}\\
$\lambda$, asymmetry, and $\mid\alpha\mid$ cuts:
\noindent\hspace{3 cm}$\lambda$ and asymmetry cuts:\\
\includegraphics[width=0.5\textwidth]{3cuts/peakIntegrals_v_pT.png}
\includegraphics[width=0.5\textwidth]{2cuts/peakIntegrals_v_pT.png}
$\lambda$ cuts:
\noindent\hspace{4 cm} No cuts at all:\\
\includegraphics[width=0.5\textwidth]{1cuts/peakIntegrals_v_pT.png}
\includegraphics[width=0.5\textwidth]{0cuts/peakIntegrals_v_pT.png}
\end{frame}

\section{Collisions over Mass Varied with Momentum for All Quantities of Cuts} % The section with the collisions vs. mass plots for every momentum interval and permutation of cuts

Each subsection below shows pion entries vs. mass graphs for the following momentum intervals, in this order from left to right: 5-7.5 GeV, 7.5-10 GeV, 10-11 GeV, 11-12 GeV, 12-13 GeV, 13-15 GeV, 15-18 GeV

\subsection{Pion Entries over Mass Varied with Momentum for $\lambda$, asymmetry, $\mid\alpha\mid$, and Ncells cuts}
\begin{frame}{}
\includegraphics[width=0.3\textwidth]{{4cuts/MyFit_Ptmin_5.00_Ptmax_7.50}.png}
\includegraphics[width=0.3\textwidth]{{4cuts/MyFit_Ptmin_7.50_Ptmax_10.00}.png}
\includegraphics[width=0.3\textwidth]{{4cuts/MyFit_Ptmin_10.00_Ptmax_11.00}.png}
\includegraphics[width=0.3\textwidth]{{4cuts/MyFit_Ptmin_11.00_Ptmax_12.00}.png}
\includegraphics[width=0.3\textwidth]{{4cuts/MyFit_Ptmin_12.00_Ptmax_13.00}.png}
\includegraphics[width=0.3\textwidth]{{4cuts/MyFit_Ptmin_13.00_Ptmax_15.00}.png}
\includegraphics[width=0.3\textwidth]{{4cuts/MyFit_Ptmin_15.00_Ptmax_18.00}.png}
\end{frame}

\subsection{Pion Entries over Mass Varied with Momentum for $\lambda$, asymmetry, and $\mid\alpha\mid$ cuts}
\begin{frame}{}
\includegraphics[width=0.3\textwidth]{{3cuts/MyFit_Ptmin_5.00_Ptmax_7.50}.png}
\includegraphics[width=0.3\textwidth]{{3cuts/MyFit_Ptmin_7.50_Ptmax_10.00}.png}
\includegraphics[width=0.3\textwidth]{{3cuts/MyFit_Ptmin_10.00_Ptmax_11.00}.png}
\includegraphics[width=0.3\textwidth]{{3cuts/MyFit_Ptmin_11.00_Ptmax_12.00}.png}
\includegraphics[width=0.3\textwidth]{{3cuts/MyFit_Ptmin_12.00_Ptmax_13.00}.png}
\includegraphics[width=0.3\textwidth]{{3cuts/MyFit_Ptmin_13.00_Ptmax_15.00}.png}
\includegraphics[width=0.3\textwidth]{{3cuts/MyFit_Ptmin_15.00_Ptmax_18.00}.png}
\end{frame}

\subsection{Pion Entries over Mass Varied with Momentum for $\lambda$ and asymmetry cuts}
\begin{frame}{}
\includegraphics[width=0.3\textwidth]{{2cuts/MyFit_Ptmin_5.00_Ptmax_7.50}.png}
\includegraphics[width=0.3\textwidth]{{2cuts/MyFit_Ptmin_7.50_Ptmax_10.00}.png}
\includegraphics[width=0.3\textwidth]{{2cuts/MyFit_Ptmin_10.00_Ptmax_11.00}.png}
\includegraphics[width=0.3\textwidth]{{2cuts/MyFit_Ptmin_11.00_Ptmax_12.00}.png}
\includegraphics[width=0.3\textwidth]{{2cuts/MyFit_Ptmin_12.00_Ptmax_13.00}.png}
\includegraphics[width=0.3\textwidth]{{2cuts/MyFit_Ptmin_13.00_Ptmax_15.00}.png}
\includegraphics[width=0.3\textwidth]{{2cuts/MyFit_Ptmin_15.00_Ptmax_18.00}.png}
\end{frame}

\subsection{Pion Entries over Mass Varied with Momentum for $\lambda$ cuts}
\begin{frame}{}
\includegraphics[width=0.3\textwidth]{{1cuts/MyFit_Ptmin_5.00_Ptmax_7.50}.png}
\includegraphics[width=0.3\textwidth]{{1cuts/MyFit_Ptmin_7.50_Ptmax_10.00}.png}
\includegraphics[width=0.3\textwidth]{{1cuts/MyFit_Ptmin_10.00_Ptmax_11.00}.png}
\includegraphics[width=0.3\textwidth]{{1cuts/MyFit_Ptmin_11.00_Ptmax_12.00}.png}
\includegraphics[width=0.3\textwidth]{{1cuts/MyFit_Ptmin_12.00_Ptmax_13.00}.png}
\includegraphics[width=0.3\textwidth]{{1cuts/MyFit_Ptmin_13.00_Ptmax_15.00}.png}
\includegraphics[width=0.3\textwidth]{{1cuts/MyFit_Ptmin_15.00_Ptmax_18.00}.png}
\end{frame}


\subsection{Pion Entries over Mass Varied with Momentum for No cuts at all}
\begin{frame}{}
\includegraphics[width=0.3\textwidth]{{0cuts/MyFit_Ptmin_5.00_Ptmax_7.50}.png}
\includegraphics[width=0.3\textwidth]{{0cuts/MyFit_Ptmin_7.50_Ptmax_10.00}.png}
\includegraphics[width=0.3\textwidth]{{0cuts/MyFit_Ptmin_10.00_Ptmax_11.00}.png}
\includegraphics[width=0.3\textwidth]{{0cuts/MyFit_Ptmin_11.00_Ptmax_12.00}.png}
\includegraphics[width=0.3\textwidth]{{0cuts/MyFit_Ptmin_12.00_Ptmax_13.00}.png}
\includegraphics[width=0.3\textwidth]{{0cuts/MyFit_Ptmin_13.00_Ptmax_15.00}.png}
\includegraphics[width=0.3\textwidth]{{0cuts/MyFit_Ptmin_15.00_Ptmax_18.00}.png}
\end{frame}

\end{document} % In future documents, you may want to add conclusions and an abstract
% In the powerpoint, you should mainly worry about section titles and graphs