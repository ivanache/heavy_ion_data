\documentclass{beamer}
\usepackage{graphicx}
% \usepackage{beamerthemesplit} // Activate for custom appearance

\title{Comparison of ALICE data with the Jet-Jet Monte-Carlo Simulation}
\author{Ivan Chernyshev}
\date{\today}

\begin{document}

\frame{\titlepage}

\section[Outline]{}
%\frame{\tableofcontents}

%\section{Introduction}
%\subsection{Overview of the Beamer Class}
\frame
{
  \frametitle{What aspects of the simulation and the data were compared}

  \begin{itemize}
  \item The mass-pion data and its fits
  \item The shape of the normalized transverse momentum $\pi^0$ spectra 
  \item The measured (mean) values of the $\pi^0$ mass and the widths (standard deviations) of the $\pi^0$ peak in the data 
  \item The cutflow of the $\gamma$ data    
  \item How data close to the border or to bad cells compares to data farther away
  \end{itemize}
}

\frame
{
	\frametitle{Mass-Pion Data Comparison: 6-8 GeV}
	\noindent\hspace{1.5 cm}ALICE data: 
	\noindent\hspace{3.0 cm} Simulation data:\\
	\includegraphics[width=0.55\textwidth]{{Pion_data/THnSparses_072417_output/gaussian/data/angle_17mrad/MyFit_Ptmin_6.00_Ptmax_8.00}.png}
	\includegraphics[width=0.55\textwidth]{{monte_carlo/background/Pion_data/gaussian/quadratic_my_code/data/angle_17mrad/MyFit_Ptmin_6.00_Ptmax_8.00}.png}
}

\frame
{
	\frametitle{Mass-Pion Data Comparison: 8-10 GeV}
	\noindent\hspace{1.5 cm}ALICE data: 
	\noindent\hspace{3.0 cm} Simulation data:\\
	\includegraphics[width=0.55\textwidth]{{Pion_data/THnSparses_072417_output/gaussian/data/angle_17mrad/MyFit_Ptmin_8.00_Ptmax_10.00}.png}
	\includegraphics[width=0.55\textwidth]{{monte_carlo/background/Pion_data/gaussian/quadratic_my_code/data/angle_17mrad/MyFit_Ptmin_8.00_Ptmax_10.00}.png}
}

\frame
{
	\frametitle{Mass-Pion Data Comparison: 10-12 GeV}
	\noindent\hspace{1.5 cm}ALICE data: 
	\noindent\hspace{3.0 cm} Simulation data:\\
	\includegraphics[width=0.5\textwidth]{{Pion_data/THnSparses_072417_output/gaussian/data/angle_17mrad/MyFit_Ptmin_10.00_Ptmax_12.00}.png}
	\includegraphics[width=0.5\textwidth]{{monte_carlo/background/Pion_data/gaussian/quadratic_my_code/data/angle_17mrad/MyFit_Ptmin_10.00_Ptmax_12.00}.png}
}

\frame
{
	\frametitle{Mass-Pion Data Comparison: 12-14 GeV}
	\noindent\hspace{1.5 cm}ALICE data: 
	\noindent\hspace{3.0 cm} Simulation data:\\
	\includegraphics[width=0.5\textwidth]{{Pion_data/THnSparses_072417_output/gaussian/data/angle_17mrad/MyFit_Ptmin_12.00_Ptmax_14.00}.png}
	\includegraphics[width=0.5\textwidth]{{monte_carlo/background/Pion_data/gaussian/quadratic_my_code/data/angle_17mrad/MyFit_Ptmin_6.00_Ptmax_8.00}.png}
}

\frame
{
	\frametitle{Mass-Pion Data Comparison: 14-16 GeV}
	\noindent\hspace{1.5 cm}ALICE data: 
	\noindent\hspace{3.0 cm} Simulation data:\\
	\includegraphics[width=0.5\textwidth]{{Pion_data/THnSparses_072417_output/gaussian/data/angle_17mrad/MyFit_Ptmin_14.00_Ptmax_16.00}.png}
	\includegraphics[width=0.5\textwidth]{{monte_carlo/background/Pion_data/gaussian/quadratic_my_code/data/angle_17mrad/MyFit_Ptmin_14.00_Ptmax_16.00}.png}
}

\frame
{
	\frametitle{Measured $\pi^0$ Masses Comparison: Raw Data}
	\includegraphics[width=1.0\textwidth]{combination/Pion_data/MeanMasses.png}	
}

\frame
{
	\frametitle{Measured $\pi^0$ Masses Comparison: Ratios}
	\includegraphics[width=1.0\textwidth]{combination/Pion_data/PionMassRatios.png}	
}

\frame
{
	\frametitle{Measured $\pi^0$ Mass Peak Widths Comparison: Raw Data}
	\includegraphics[width=1.0\textwidth]{combination/Pion_data/MassWidths.png}	
}

\frame
{
	\frametitle{Measured $\pi^0$ Mass Peak Widths Comparison: Ratios}
	\includegraphics[width=1.0\textwidth]{combination/Pion_data/PionMassWidthRatios.png}	
}
\frame
{
	\frametitle{Shape of $\pi^0$-pT Distribution Comparison}
	\includegraphics[width=1.0\textwidth]{combination/Pion_data/NumOfPionsGraph.png}\\
	We are investigating the discrepancy between the distributions.
}

\frame
{
	\frametitle{$\pi^0 data$: Conclusions}
	\begin{itemize}
	\item Fits to the mass-$\pi^0$ data: the simulation data probably has too-small error bars, as seen in the graphs, where the slightest shift in the points can make the p-value practically equal to zero
	\item Measured $\pi^0$ Masses over pT: MC reproduces the shape but there is an offset, varying from 2 to 4 $\%$ based on pT
	\item $\pi^0$ Mass Peak Width: The mass peak is about 10$\%$ narrower for the simulation than for the ALICE data, except at the highest pT bin, where it is about 10 \% wider
	\item Shape of the distribution: for the simulation the data is skewed towards lower momenta than is the ALICE data.
	\end{itemize}
}


\frame 
{ 
\frametitle{$\gamma$ Data Cuflow: Pt 10-50 GeV, Jet-Jet Monte-Carlo simulation} 
\begin{table} 
\caption{How cuts affect number of photons} 
\centering 
\begin{tabular}{c c c c} 
\hline\hline 
Cuts & Percentage of Previous\\ [0.5ex] 
\hline
None & n/a\\
+$0.1 < \lambda <$ 0.4 & 29.0 $\%$ \\
+dR $> 20$ mrad & 54.7 $\%$ \\
+DisToBorder $>$ 0 & 92.5 $\%$ \\
+DisToBadCell $>$ 1 & 74.6 $\%$ \\
+Ncells $> 1$ & 100.0 $\%$ \\
+Exoticity $< 0.97$ & 99.9 $\%$ \\
+$|Time| < 30$ ns & n/a (time is not well-modeled)  \\
+isolation $< 4 GeV/c$ & 21.4 $\%$ \\
[1ex] 
\hline 
\end{tabular} 
\label{table:nonlin} 
\end{table} 
 Final value's portion of the original: 2.3\%
 } 

\frame 
{ 
\frametitle{$\gamma$ Data Cutflow: Pt 10-50 GeV, ALICE data} 
\begin{table} 
\caption{How cuts affect number of photons} 
\centering 
\begin{tabular}{c c c c} 
\hline\hline 
Cuts & Percentage of Previous\\ [0.5ex] 
\hline
None & n/a\\
+$0.1 < \lambda <$ 0.4 & 57.9 $\%$ \\
+dR $> 20$ mrad & 88.8 $\%$ \\
+DisToBorder $>$ 0 & 92.8 $\%$ \\
+DisToBadCell $>$ 1 & 78.4 $\%$ \\
+Ncells $> 1$ & 100.0 $\%$ \\
+Exoticity $< 0.97$ & 99.9 $\%$ \\
+$|Time| < 30$ ns & 99.8 $\%$ \\
+isolation $< 4 GeV/c$ & 56.9 $\%$ \\
[1ex] 
\hline 
\end{tabular} 
\label{table:nonlin} 
\end{table} 
 Final value's portion of the original: 21.2\%
 } 

\frame
{
	\frametitle{$\gamma$ Data Cutflow: pT-$\gamma$ spectrum for no cuts}
	\noindent\hspace{1.5 cm}ALICE data: 
	\noindent\hspace{3.0 cm} Simulation data:\\
	\includegraphics[width=0.55\textwidth]{Direct_Photon_data/cutflow/pT_photon_0_cuts.png}
	\includegraphics[width=0.55\textwidth]{monte_carlo/background/Direct_Photon_data/cutflow/pT_photon_0_cuts.png}
}

\frame
{
	\frametitle{$\gamma$ Data Cutflow: pT-$\gamma$ spectrum: added $0.1 < \lambda <$ 0.4 cut}
	\noindent\hspace{1.5 cm}ALICE data: 
	\noindent\hspace{3.0 cm} Simulation data:\\
	\includegraphics[width=0.55\textwidth]{Direct_Photon_data/cutflow/pT_photon_1_cuts.png}
	\includegraphics[width=0.55\textwidth]{monte_carlo/background/Direct_Photon_data/cutflow/pT_photon_1_cuts.png}
}

\frame
{
	\frametitle{$\gamma$ Data Cutflow: pT-$\gamma$ spectrum: added dR $> 20$ mrad cut}
	\noindent\hspace{1.5 cm}ALICE data: 
	\noindent\hspace{3.0 cm} Simulation data:\\
	\includegraphics[width=0.55\textwidth]{Direct_Photon_data/cutflow/pT_photon_2_cuts.png}
	\includegraphics[width=0.55\textwidth]{monte_carlo/background/Direct_Photon_data/cutflow/pT_photon_2_cuts.png}
}

\frame
{
	\frametitle{$\gamma$ Data Cutflow: pT-$\gamma$ spectrum: added DisToBorder $>$ 0 cut}
	\noindent\hspace{1.5 cm}ALICE data: 
	\noindent\hspace{3.0 cm} Simulation data:\\
	\includegraphics[width=0.55\textwidth]{Direct_Photon_data/cutflow/pT_photon_3_cuts.png}
	\includegraphics[width=0.55\textwidth]{monte_carlo/background/Direct_Photon_data/cutflow/pT_photon_3_cuts.png}
}

\frame
{
	\frametitle{$\gamma$ Data Cutflow: pT-$\gamma$ spectrum: added DisToBadCell $>$ 1 cut}
	\noindent\hspace{1.5 cm}ALICE data: 
	\noindent\hspace{3.0 cm} Simulation data:\\
	\includegraphics[width=0.55\textwidth]{Direct_Photon_data/cutflow/pT_photon_4_cuts.png}
	\includegraphics[width=0.55\textwidth]{monte_carlo/background/Direct_Photon_data/cutflow/pT_photon_4_cuts.png}
}

\frame
{
	\frametitle{$\gamma$ Data Cutflow: pT-$\gamma$ spectrum: added Ncells $> 1$}
	\noindent\hspace{1.5 cm}ALICE data: 
	\noindent\hspace{3.0 cm} Simulation data:\\
	\includegraphics[width=0.55\textwidth]{Direct_Photon_data/cutflow/pT_photon_5_cuts.png}
	\includegraphics[width=0.55\textwidth]{monte_carlo/background/Direct_Photon_data/cutflow/pT_photon_5_cuts.png}
}

\frame
{
	\frametitle{$\gamma$ Data Cutflow: pT-$\gamma$ spectrum: added Exoticity $< 0.97$ cut}
	\noindent\hspace{1.5 cm}ALICE data: 
	\noindent\hspace{3.0 cm} Simulation data:\\
	\includegraphics[width=0.55\textwidth]{Direct_Photon_data/cutflow/pT_photon_6_cuts.png}
	\includegraphics[width=0.55\textwidth]{monte_carlo/background/Direct_Photon_data/cutflow/pT_photon_6_cuts.png}
}

\frame
{
	\frametitle{$\gamma$ Data Cutflow: pT-$\gamma$ spectrum: added $|Time| < 30$ ns cut}
	\noindent\hspace{1.5 cm}ALICE data: 
	\noindent\hspace{3.0 cm} Simulation data:\\
	\includegraphics[width=0.55\textwidth]{Direct_Photon_data/cutflow/pT_photon_7_cuts.png}
	\includegraphics[width=0.55\textwidth]{monte_carlo/background/Direct_Photon_data/cutflow/pT_photon_7_cuts.png}
}

\frame
{
	\frametitle{$\gamma$ Data Cutflow: pT-$\gamma$ spectrum: added isolation $< 4$ GeV/c cut}
	\noindent\hspace{1.5 cm}ALICE data: 
	\noindent\hspace{3.0 cm} Simulation data:\\
	\includegraphics[width=0.55\textwidth]{Direct_Photon_data/cutflow/pT_photon_8_cuts.png}
	\includegraphics[width=0.55\textwidth]{monte_carlo/background/Direct_Photon_data/cutflow/pT_photon_8_cuts.png}
}

\frame
{
	\frametitle{$\gamma$ cutflow: Conclusions}
	\begin{itemize}
	\item $\gamma$ cutflow: The monte-carlo simulation had an order of magnitude less data left after the cuts than did the ALICE data, which makes sense since the simulation is intended to measure background
	\item This discrepancy is most noticeably present in the $\lambda_{02}$, dR, and isolation cuts
	\end{itemize}
}


\frame
{
	\frametitle{ALICE data: $\gamma$ spectrum ratios of various distance to border and distance to bad cell bins}
	\includegraphics[width=0.55\textwidth]{Direct_Photon_data/cut_comparisons/0.1<lambda<0.4/4<pT<20/photon_ratios_distoborder.png}
	\includegraphics[width=0.55\textwidth]{Direct_Photon_data/cut_comparisons/0.1<lambda<0.4/4<pT<20/photon_ratios_distobadcells.png}
}

\frame
{
	\frametitle{Simulation data: $\gamma$ spectrum ratios of various distance to border and distance to bad cell bins}
	\includegraphics[width=0.55\textwidth]{monte_carlo/background/Direct_Photon_data/cut_comparisons/0.1<lambda<0.4/5<pT<20/photon_ratios_distoborder.png}
	\includegraphics[width=0.55\textwidth]{monte_carlo/background/Direct_Photon_data/cut_comparisons/0.1<lambda<0.4/5<pT<20/photon_ratios_distobadcells.png}
}

\frame
{
	\frametitle{Distance to Border and Distance to Bad Cell: Conclusions}
	\begin{itemize}
	\item Distance to Border and Distance to Bad Cell spectrum ratios: Just like in the ALICE data, in the simulation the ratios for distance to border = 0 and distance to bad cell = 1 are farther from unity than the rest of the ratios. 
	\item However, due to the low error bars, it is unknown whether or not any of the ratios are close enough to unity to be counted as unity, though visually it does seem to be the case that all bins for the simulation except distance to border = 0 have a ratio essentially equal to unity.
	\end{itemize}
}

\frame 
{
	\frametitle{For reference: the Github repository with all of the code for my data}
	\includegraphics[width=1.0\textwidth]{github_image.png}
}

\end{document}
